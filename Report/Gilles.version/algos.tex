% Sur MacCJS, faire
%/usr/local/texlive/2015/bin/x86_64-darwin/pdflatex
%
\documentclass[]{article}
\usepackage[utf8]{inputenc}
\usepackage[T1]{fontenc}
\usepackage[frenchb]{babel}
\usepackage{listings}
\usepackage{amsmath}
\usepackage[pdftex]{hyperref}
\usepackage{graphicx}
\usepackage[ruled]{algorithm}
\usepackage{algorithmicx}
%\usepackage{algorithm,algorithmicx,algpseudocode}
\usepackage{algpseudocode}
\usepackage{amssymb}
\usepackage{mathabx}
\usepackage{dsfont}
\usepackage{enumitem}
\usepackage{color}
\newcommand{\be}{\begin{eqnarray*}}
\newcommand{\ee}{\end{eqnarray*}}
\newcommand{\dsp}{\displaystyle}
\algrenewcommand{\algorithmiccomment}[1]{\hfill \# {\scriptsize #1}} 
\algnewcommand\algorithmicinput{\textbf{Input:}}
\algnewcommand\Input{\item[\algorithmicinput]}
\algnewcommand\algorithmicoutput{\textbf{Output:}}
\algnewcommand\Output{\item[\algorithmicoutput]}
\algnewcommand\algodesc{\textbf{Description:}}
\algnewcommand\Desc{\item[\algodesc]}
\algnewcommand\algouse{\textbf{Usage:}}
\algnewcommand\Use{\item[\algouse]}
\makeatletter
\newcommand{\algrule}[1][.2pt]{\par\vskip.5\baselineskip\hrule height #1\par\vskip.5\baselineskip}
\makeatother

\def\R{\mathbb{R}}
\def\UN{\mathds{1}}
\def\Ik{\mathbb{I}_k}
\def\Ikm{\mathbb{I}_{\scriptscriptstyle{k-1}}}
\def\Il{\mathbb{I}_l}
\def\Im{\mathbb{I}_m}
\def\Tmax{T^{\max}}
\def\ver{V1}  %version

\begin{document}
\shorthandoff{:}
\tableofcontents
\listofalgorithms

\section{Notation}
%%%
% Ik
%%%
We denote by $K$ the number of intervals (partitions) of the time interval $(0,\Tmax)$, each time interval having a size $\delta$ (most of the time we have $K\,\delta=\Tmax$).

We introduce the following notation for $0\leq k \leq K-1$ ($\Ik$ is an open interval on the left and closed on the right) to have a shorter form
\be
\Ik&=&(k\,\delta,(k+1)\,\delta]\\
(t-\Ik)&=&[t-(k+1)\,\delta,\,t-k\,\delta) \ \ \forall t
\ee
%%%
% Ensemble des spikes
%%% 

We assume that $(0,\Tmax)$ corresponds to the union of the partitions (no overlapping).
\be
(0,\Tmax)=\bigcup_{k=0}^{K-1} \Ik
\ee
Numbering of partitions starts from 0!
In Matlab or in R, indices start from 1.
As we take into account $\mu$ index, index $k+2$ in $b$ corresponds to the k-th partition.

We choose $\gamma$ equal to 3, in the computation of $d$

We denote by $\varphi_k(t)$ or by $N_{[t-k\delta,t-(k-1)\delta)}$ (or by $N_{t-\Ikm}$) the number of spikes of the point process in the interval $[t-k\delta,t-(k-1)\delta)$ (or $t-\Ikm$).

For $1\leq k \leq K$
\begin{eqnarray}
\begin{array}{lll}
\varphi_k(t)&=&N_{[t-k\delta,T-(k-1)\delta)}=N_{t-\Ikm}\\
\varphi_k(t)&=&\hbox{number of spikes of the point process in the interval }(t-\Ikm)
\end{array}
\end{eqnarray}

$\varphi_k$ is a piecewise constant function. Représenter un exemple
\section{Matrix computation}
We will compute here vectors and matrices 

For the sake of simplicity, we denote by $T_i$ the spike times contained in the spike array.
%%%%%
%def de A
%%%%%
\be
A_{ij} = m \hbox{ iff } (T_j - T_i) \in \Im \Leftrightarrow m\,\delta < T_j-T_i \leq (m+1)\delta
\ee
%%%%%
%def de b
%%%%%
\be
b_{k}=\int_0^{Tmax} \psi_t(\varphi_k)\,dN_t \ \forall k \in \ldbrack 0,K-1\ldbrack
\ee

Deterministic way.


We have, for $k\geq 1$
\begin{eqnarray}
\label{eq:defb}
\begin{array}{lll}
b_{k}&=&\dsp\int_0^{Tmax} \varphi_k(t) \,dN_t=\dsp\int_0^{Tmax} N_{[t-k\delta,t-(k-1)\delta)} \,dN_t=\dsp\int_0^{Tmax} N_{t-\Ikm} \,dN_t\\[8pt]
b_{k}&=&\dsp\int_0^{Tmax} \varphi_k(t) \,dN_t= \dsp\int\sum_{T<\Tmax} N_{[T-k\delta,T-(k-1)\delta)}= \dsp\sum_{T<\Tmax} N_{T-\Ikm}
\end{array}
\end{eqnarray}


%%%%%
%def de d
%%%%%
\be
d_k=\sqrt{2\gamma\log(\Tmax)\,\int_0^{\Tmax}\psi_t^2(\varphi_k)\,dN_t}+\dsp\frac{\gamma\,\log(\Tmax)}{3}\sup_{t\in[0,\Tmax]}|\psi_t(\varphi_k)|
\ee

Deterministic way
\be
d_k=\sqrt{2\gamma\log(\Tmax)\,\int_0^{\Tmax}\varphi_k^2(t)\,dN_t}+\dsp\frac{\gamma\,\log(\Tmax)}{3}\sup_{t\in[0,\Tmax]}|\varphi_k(t))|
\ee


%%%%%
%def de G
%%%%%
\be
G_{kl}=\int_0^{Tmax} \psi_t(\varphi_k)\,\psi_t(\varphi_l)\,\,dt \ \ \forall k,l \in \ldbrack 0,K-1\ldbrack
\ee

Deterministic way
\be
G_{kl}=\int_0^{Tmax} \sum_{T_i,T_j<t}\mathds{1}_{\mathbb{I}_k}(t-T_i)\,\mathds{1}_{\mathbb{I}_l}(t-T_j) \forall k,l \in \ldbrack 0,K-1\ldbrack
\ee


\begin{algorithm}[!h]
\caption{Partial computation of $A$ and $b$: countpair function}
\begin{algorithmic}[1] 
%Algorithm
%Step 1: Find the first index j such that T¡ > 0, calls jstart.
%Step 2: From each index j from jstart, we 'look' back II¿ by define Tlow, and Tup.
%Step 3: If we flnd the index ¿ such lhat T¿ belongs to the interval (Tlow,Tup], increase
%count, and record it in the matrix ,4. \Me stop immediately when T¿ < Tlota.

\Use countpair(spike, del, k, A)
  \Desc Partial computation of $A$ and computation of $b$ for the k-th partition
  \Input spike; array of spike times
\Input del; $\delta$, delay
\Input k; partition index ($k \in \ldbrack 0,K-1 \rdbrack$)
\Input A; $N$-by-$N$ matrix already initialized\Comment{$N$ is the number of spikes}
\Output Returns $A$ and count
\algrule
% \\\hrulefill
\State count $\gets$ 0 \Comment{count corresponds to b[k+2]}
\State N $\gets$ length(spike) \Comment{$N$ is the number of spikes}
\State jstart $\gets$ which(spike>0)[1] \Comment{first index such that spike > 0}
\For{j=jstart:N}
\State $T_{low}$ $\gets$ spike[j] - (k+1)*del
\State $T_{up}$ $\gets$ spike[j] - k*del
\State i $\gets$ j-1
\While{((i>0) \& (i<N))}
\If{spike[i] $\in$ $[T_{low},\,T_{up})$}
\State count $\gets$ count + 1
\State A[i,j] $\gets$ k
\Else
\If{(spike[i] $<T_{low}$)}
\State break
\EndIf
\EndIf
\EndWhile
\EndFor
\State \Return list(count, A)
\end{algorithmic}
\end{algorithm}

%%%%
% algo de A
%%%

\begin{algorithm}[!h]
\caption{Full computation of $A$}
\begin{algorithmic}[1] 
\Use computeA(spike, del, K)
  \Desc Full computation of $A$ using \textit{countpair} function
  \Input spike; array of spike times
\Input del; $\delta$
  \Input K; number of partitions
\Output A; $N$-by-$N$ matrix \Comment{$N$ is the number of spikes} (K+1) (see \ref{eq:defb})
\algrule
% \\\hrulefill
\State N $\gets$ length(spike)
\State A $\gets$ matrix(-2, nrow=N, ncol=N) \Comment{initialization}
\For{k=0:(K-1)}
\State A $\gets$ countpair(spike, del, k, A)[[2]]
\EndFor
\State \Return A
\end{algorithmic}
\end{algorithm}

%%%%
% algo de b
%%%

\begin{algorithm}[!h]
\caption{Full computation of $b$}
\begin{algorithmic}[1] 
\Use computeb\_\ver(spike, del, Tmax)
  \Desc Full computation of $b$ using \textit{countpair} function
  \Input spike; array of spike times
\Input del; $\delta$
  \Input K; number of partitions
\Output b; (K+1) vector (see \ref{})
\algrule
% \\\hrulefill
\State K $\gets$ floor(Tmax/del)
\State N $\gets$ length(spike)
\State b $\gets$ vector(mode='double', length=K+1) \Comment{initialization} 
\State jstart $\gets$ which(spike>0)[1] \Comment{first index such that spike > 0}
\State b[1] $\gets$ N \Comment{{\color{red}{A verifier: c'est le nb de spikes dans (0,$\Tmax$), donc pas forcément N}}}
\If{jstart>0}
\For{k=0:(K-1)} 
\For{j=jstart:N} \Comment{{\color{red}{to improve for jstart}}}
\State $T_{low}$ $\gets$ spike[j] - (k+1)*del
\State $T_{up}$ $\gets$ spike[j] - k*del
\State i $\gets$ j-1
\While{((i>0) \& (i<N))}
\If{(spike[i] $\in$ [$T_{low}$, $T_{up}$))}
\State b[k+2] $\gets$ b[k+2] + 1
\Else
\If{($T_{low}>$spike[i])}
\State break
\EndIf
\EndIf
\State i $\gets$ i-1
\EndWhile
\EndFor
\EndFor
\EndIf
\State \Return b
\end{algorithmic}
\end{algorithm}

%%%%
% algo de b
%%%

\begin{algorithm}[!h]
\caption{Full computation of $b$}
\begin{algorithmic}[1] 
\Use computeb\_\ver(spike, del, Tmax)
  \Desc Full computation of $b$ using \textit{countpair} function
  \Input spike; array of spike times
\Input del; $\delta$
  \Input K; number of partitions
\Output b; (K+1) vector (see \ref{})
\algrule
% \\\hrulefill
\State N $\gets$ length(spike)
\State b $\gets$ matrix(0, nrow=N, ncol=N) \Comment{initialization} 
\State A $\gets$ matrix(-2, nrow=N, ncol=N) \Comment{A is not important for b but needs to be initialized}
\State b[1] $\gets$ N
\For{k=0:(K-1)}
\State b[k+2] $\gets$ countpair(spike, del, k, A)[[1]]
\EndFor
\State \Return b
\end{algorithmic}
\end{algorithm}





%%%%
% algo de d
%%%

\begin{algorithm}[!h]
\caption{Partial computation of $d$}
\begin{algorithmic}[1] 
%Algorithm
%Step 1: Find the first index j such that T¡ > 0, calls jstart.
%Step 2: From each index j from jstart, we 'look' back II¿ by define Tlow, and Tup.
%Step 3: If we flnd the index ¿ such lhat T¿ belongs to the interval (Tlow,Tup], increase
%count, and record it in the matrix ,4. \Me stop immediately when T¿ < Tlota.

\Use compute\_d(spike, del, Tmax, A, k, gamma)
  \Desc Partial computation of $A$ and computation of $b$ for the k-th partition
  \Input spike; array of spike times
\Input del; $\delta$, delay
\Input $\Tmax$; maximum time of the experience
\Input A; $N$-by-$N$ matrix already initialized\Comment{$N$ is the number of spikes}
\Input k; partition index ($k \in \ldbrack 0,K-1 \rdbrack$)
\Input $\gamma$; constant used in the definition of $d$
\Output Returns $d$
\algrule
% \\\hrulefill
\State iind $\gets$ which(A==k, arr.ind=TRUE) \Comment{iind is a matrix}
\State z $\gets$ vector(length=length(spike), mode='numeric') \Comment{for maximum value}
\For{i=1:N} \Comment{$N\geq 1$!}
\State  $T_1 \gets spike[i] + k*del$; $T_2 = spike[i] + (k+1)*del$
\If{($T_2 \leq \Tmax$) \& ($T_2 \geq 0$)}
\State z[i] $\gets$  countspike(spike, del, k, $T_2$)
\EndIf
\If{($T_1 < \Tmax$) \& ($T_1 \geq 0$)}
\State z[i] $\gets$  $\max$(z[i], 1)
\EndIf
\If{($T_1 < 0$) \& ($T_2> \Tmax$)}
\State z[i] $\gets$  $\max$(z[i], 1)
\EndIf
\EndFor
\State $d$ $\gets$ $\sqrt{2\gamma\log(\Tmax)\,count\_conse(iind[,2])}+\gamma \log(\Tmax)/3* \max(z) $
\State \Return $d$
\end{algorithmic}
\end{algorithm}

\begin{algorithm}[!h]
\caption{count\_conse}
\begin{algorithmic}[1] 
\Use count\_conse(x)
  \Desc Computation of consecutive ?
  \Input x; array of ?
\Output res; 
\algrule

\State count $\gets$ 0
\State conse $\gets$ 1
\State i $\gets$ 1
\State n $\gets$ length(x)
\If{(!n)}
\State \Return 0
\EndIf
\State compare $\gets$ x[1]
\State x[n+1] $\gets$ -1 \Comment{length of x is increased}
\While{($i\leq n$)}
\If{(x[i+1]==compare)}
\State conse $\gets$ conse + 1
\Else
\State count $\gets$ count + conse*(conse-1)
\State conse $\gets$ 1
\State compare $\gets$ x[i+1]
\EndIf
\State i $\gets$ i+1
\EndWhile
\State res $\gets$ n+ count
\State \Return res
\end{algorithmic}
\end{algorithm}

\begin{algorithm}[!h]
\caption{countspike}
\begin{algorithmic}[1] 
\Use countspike(spike, del, k, t)
  \Desc Computation of the numbers of spikes such that ?
  \Input spike; array of spike times
\Input del; $\delta$
\Input k; partition index
\Input t; spike time
\Output resuk;
\algrule
\State resuk $\gets$ 0; N $\gets$ length(spike); $\varepsilon$ $\gets$ 1e-12
\For{(i=1:N)} \Comment{N should be $\geq 1$!}
\If{(spike[i]$\geq$ (t-(k+1)*del)) \& (spike[i]< (t-k*del-$\varepsilon$))} \Comment{$\varepsilon$ for numerical errors}
\State resuk $\gets$ resuk + 1
\EndIf
\EndFor
\State \Return resuk
\end{algorithmic}
\end{algorithm}


%%%%
% algo de b
%%%

\begin{algorithm}[!h]
\caption{Full computation of $d$}
\begin{algorithmic}[1] 
\Use computed(spike, del, K)
  \Desc Full computation of $d$ using \textit{count\_conse} and \textit{countspike} functions
  \Input spike; array of spike times
\Input del; $\delta$
  \Input K; number of partitions
\Output d; (K+1) vector (see \ref{}) % equation de d
\algrule
% \\\hrulefill
\State N $\gets$ length(spike)
\State d $\gets$ vector(length=K+1, mode='numeric') \Comment{initialization} 
\If{missing(A)} \Comment{one needs to compute $A$ - see }
\State A $\gets$ matrix(-2, nrow=N, ncol=N) 
\For{k=0:(K-1)}
\State A $\gets$ countpair(spike, del, k, A)[[2]]
\EndFor
\EndIf
\State d[1] $\gets\sqrt{2\,\gamma\,\log{\Tmax}\,N}+\gamma\,\log{\Tmax}/3.$
\For{k=0:(K-1)}
\State d[k+2] $\gets$ compute\_d(spike, del, $\Tmax$, A, k, gamma)
\EndFor
\State \Return b
\end{algorithmic}
\end{algorithm}






%%%%
% algo de G
%%%

\begin{algorithm}[!h]
\caption{Full computation of $G$}
\begin{algorithmic}[1] 
\Use computeG\_V1(spike, delta, Tmax, A)
  \Desc Full computation of $G$
  \Input spike; array of spike times
\Input delta; $\delta$
\Input $\Tmax$; final time
\Input A; matrix 
\Output G; (K+1)-by-(K+1) matrix (see \ref{}) % equation de d
\algrule
% \\\hrulefill
\State N $\gets$ length(spike)
\State K $\gets$ floor(Tmax/delta)
\If{missing(A)} \Comment{one needs to compute $A$ - see }
\State A $\gets$ computeA(spike, delta, K)
\EndIf
\State G $\gets$ matrix(0, nrow=K+1, ncol=K+1) \Comment{initialization}
\If{K$>2$}
\For{l=0:(K-2)} \Comment{computation of the lower part of G}
\For{k=(l+1):(K-1)}
\State rescase1 $\gets$ 0; rescase2 $\gets$ 0
\Comment{1st case}
\State iind $\gets$ which(A==(k-l), arr.ind=TRUE) \Comment{matrix indices}
\State lidex $\gets$ nrow(iind)
\For{idex=1:lidex} \Comment{lidex needs to be $>0$}
\State inter\_fi $\gets$ $\min(spike[iind[idex,1]]+(k+1)*\delta,\Tmax)$
\State inter\_de $\gets$ $\max(spike[iind[idex,2]]+l*\delta,0)$
\State length\_inter $\gets$ inter\_fi - inter\_de
\If{length\_inter>0}
\State rescase1 $\gets$ rescase1 + length\_inter
\EndIf
\EndFor
\State iind $\gets$ which(A==(k-l-1), arr.ind=TRUE) \Comment{matrix indices}
\State lidex $\gets$ nrow(iind)
\For{idex=1:lidex} \Comment{lidex needs to be $>0$}
\State inter\_fi $\gets$ $\min(spike[iind[idex,1]]+(k+1)*\delta,\Tmax)$
\State inter\_de $\gets$ $\max(spike[iind[idex,2]]+l*\delta,0)$
\State length\_inter $\gets$ inter\_fi - inter\_de
\If{length\_inter>0}
\State rescase1 $\gets$ rescase1 + length\_inter
\EndIf
\EndFor
\algstore{algG}
\end{algorithmic}
\end{algorithm}

\begin{algorithm}[!h]
%\caption{Full computation of $G$}
\begin{algorithmic}[1] 
\algrestore{algG}
\State iind $\gets$ which(A==(k-l-1), arr.ind=TRUE) \Comment{2nd case, iind contains matrix indices}
\State lidex $\gets$ nrow(iind)
\For{idex=1:lidex} \Comment{lidex needs to be $>0$}
\State inter\_fi $\gets$ $\min(spike[iind[idex,2]]+(l+1)*\delta,\Tmax)$
\State inter\_de $\gets$ $\max(spike[iind[idex,1]]+k*\delta,0)$
\State length\_inter $\gets$ inter\_fi - inter\_de
\If{length\_inter>0}
\State rescase2 $\gets$ rescase2 + length\_inter
\EndIf
\EndFor
\State G[k+2,l+2] $\gets$ rescase1 + rescase2
\EndFor
\EndFor
\For{k=0:(K-1)} \Comment{$k=l$}
\State rescase1 $\gets$ 0
\State iind $\gets$ which(A==0, arr.ind=TRUE) \Comment{matrix indices}
\State lidex $\gets$ nrow(iind)
\For{idex=1:lidex} \Comment{lidex needs to be $>0$}
\State inter\_fi $\gets$ $\min(spike[iind[idex,1]]+(k+1)*\delta,\Tmax)$
\State inter\_de $\gets$ $\max(spike[iind[idex,2]]+k*\delta,0)$
\State length\_inter $\gets$ inter\_fi - inter\_de
\If{length\_inter>0}
\State rescase1 $\gets$ rescase1 + length\_inter
\EndIf
\EndFor
\State rescase3 $\gets$ 0
\For{idex=1:N} \Comment{N should be $>0$!}
\State inter\_fi $\gets$ $\min(spike[i]+(k+1)*\delta,\Tmax)$
\State inter\_de $\gets$ $\max(spike[i]+k*\delta,0)$
\State length\_inter $\gets$ inter\_fi - inter\_de
\If{length\_inter>0}
\State rescase3 $\gets$ rescase3 + length\_inter
\EndIf
\EndFor
\State G[k+2,k+2] $\gets$ 2*rescase1 + rescase3
\EndFor
\Comment{1st column of $G$}
\algstore{algG1}
\end{algorithmic}
\end{algorithm}

\begin{algorithm}[!h]
%\caption{Full computation of $G$}
\begin{algorithmic}[1] 
\algrestore{algG1}
\State G[1,1] $\gets \Tmax$
\For{k=0:(K-1)}
\State rescase $\gets$ 0
\For{i=1:N} \Comment{N should be $>0$!}
\State inter\_fi $\gets$ $\min(spike[i]+(k+1)*\delta,\Tmax)$
\State inter\_de $\gets$ $\max(spike[i]+k*\delta,0)$
\State length\_inter $\gets$ inter\_fi - inter\_de
\If{length\_inter>0}
\State rescase $\gets$ rescase + length\_inter
\EndIf
\EndFor
\State G[k+2,1] $\gets$ rescase
\EndFor
\EndIf
\State  lowG $\gets$ lower.tri(G)  \Comment{to get the strictly lower triangular part}
\State  G[upper.tri(G)] $\gets$ G[lowG][order(row(G)[lowG], col(G)[lowG])] \Comment{to make G symmetric}
\State \Return G
\end{algorithmic}
\end{algorithm}


%complexity


\section{Ordinary Least Squares}
\section{Lasso shooting algorithm}
\begin{algorithm}[!h]
\caption{Lasso shooting}
\begin{algorithmic}[1] 
%Algorithm
%Step 1: Find the first index j such that T¡ > 0, calls jstart.
%Step 2: From each index j from jstart, we 'look' back II¿ by define Tlow, and Tup.
%Step 3: If we flnd the index ¿ such lhat T¿ belongs to the interval (Tlow,Tup], increase
%count, and record it in the matrix ,4. \Me stop immediately when T¿ < Tlota.

\Use countpair(T, $\delta$, k, A)
  \Desc Computation of $A$
  \Input T; array of spike times
\Input $\delta$; delay
  \Input k; number of partitions
\Input A; $N$-by-$N$ matrix \Comment{$N$ is the number of spikes}
\Output Returns $A$ after counting
\algrule
% \\\hrulefill
\State initialization of $a$
\State m $\gets$ 0
\State $a^{old}$ $\gets$ $a$
\While{($|F(a)-F(a^{old})|>\varepsilon$)}
\For{i=0,K}
\State $a_i$ $\gets$ 
\State m $\gets$ m + 1
\EndFor
\EndWhile
\State \Return $a$
\end{algorithmic}
\end{algorithm}

\section{Active set}
\end{document}
