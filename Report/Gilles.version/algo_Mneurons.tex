%
% Se compile en utilisant le Makefile (sage inside)
%
\documentclass{article}
\usepackage{graphicx}
\usepackage{amssymb,amsmath}
\usepackage{hyperref}
\usepackage[latin1]{inputenc}
\usepackage{enumitem}
%\usepackage{sagetex}
\usepackage{mathabx}
\usepackage{dsfont}
\usepackage[ruled]{algorithm}
\usepackage{algorithmicx}
\usepackage[noend]{algpseudocode}
\algrenewcommand{\algorithmiccomment}[1]{\hfill \# {\scriptsize #1}} 
\algnewcommand\algorithmicinput{\textbf{Input:}}
\algnewcommand\Input{\item[\algorithmicinput]}
\algnewcommand\algorithmicoutput{\textbf{Output:}}
\algnewcommand\Output{\item[\algorithmicoutput]}
\algnewcommand\algodesc{\textbf{Description:}}
\algnewcommand\Desc{\item[\algodesc]}
\algnewcommand\algouse{\textbf{Usage:}}
\algnewcommand\Use{\item[\algouse]}
\newcommand{\be}{\begin{eqnarray*}}
\newcommand{\ee}{\end{eqnarray*}}
\newcommand{\ben}{\begin{eqnarray}}
\newcommand{\een}{\end{eqnarray}}
\newcommand{\dsp}{\displaystyle}
\makeatletter
\newcommand{\algrule}[1][.2pt]{\par\vskip.5\baselineskip\hrule height #1\par\vskip.5\baselineskip}
\makeatother

\DeclareMathOperator\erf{erf}
\def\E{\mathbb{E}}
\def\R{\mathbb{R}}
\def\mzero{\widehat{m}_0}
\def\vsq{\widehat{v}^2}
\def\P{\mathbb{P}}
\def\UN{\mathds{1}}
\def\Ik{\mathbb{I}_k}
\def\Ikun{\mathbb{I}_{k_1}}
\def\Ikdeux{\mathbb{I}_{k_2}}
\def\Il{\mathbb{I}_l}
\def\Im{\mathbb{I}_m}
\def\Tmin{T_{\min}}
\def\Tmax{T^{\max}}
\def\Ikm{\mathbb{I}_{\scriptscriptstyle{k-1}}}
\def\Nltk{N^{(l)}_{[t-k\delta,t-(k-1)\delta[}}
\def\NlTk{N^{(l)}_{[T-k\delta,T-(k-1)\delta[}}
\def\Nluntkun{N^{(l_1)}_{[t-k_1\delta,t-(k_1-1)\delta[}}
\def\Nldeuxtkdeux{N^{(l_2)}_{[t-k_2\delta,t-(k_2-1)\delta[}}
\title{Lasso for Hawkes process}

\begin{document}
\maketitle
\date
    \section{Notation and definition (may be imprecise)}
\begin{itemize}
  \item $M$, total number of neurons
\item $K$, number of bins
  \item $\delta$, size
\item $Ntot$, total number of spikes
\item $\Ik=]{(k-1)}\,\delta,\:k\,\delta]=({(k-1)}\,\delta,\:k\,\delta], \ \ \forall\,k \,\in \ldbrack 1, K\rdbrack$
\item for any $\theta$, $k \in \ldbrack 1, K\rdbrack$, $\theta+\Ik$ is an interval such that $$ \theta+\Ik=(\theta+\Ik)=]\theta+(k-1)\delta,\theta+k\,\delta]=(\theta+(k-1)\delta,\theta+k\,\delta]$$
\item for any $\theta$, $k \in \ldbrack 1, K\rdbrack$, $\theta-\Ik$ is an interval such that $$\theta-\Ik=(\theta-\Ik)=[\theta-k\,\delta,\theta-(k-1)\,\delta[=[\theta-k\,\delta,\theta-(k-1)\,\delta)$$
  \item $A$ is the scope, such that $A=K\,\delta$ and a maximum length. If the distance between two spikes is strictly greater than $A$, they don't influence each other
  \item $\Tmin$ is the  beginning of the study time interval
  \item $\Tmax$ is the end of the study time interval
\item $Int(x)=\lfloor x \rfloor$ represents the integer part of $x$
  \item $\lambda^{(r)}$ is the functional intensity of neuron $r$, function depending on time
\item $\mu^{(r)}$ is the spontaneous part of $\lambda^{(r)}$
\item $\mathcal{N}^{(l)}$ is the set of spike values of neuron $l$
\item $N^{(l)}_{t-\Ik}=N^{(l)}_{[t-k\delta,t-(k-1)\delta[}$ represents the number of spikes $\theta$ of $\mathcal{N}^{(l)}$ such that $$\theta\in (t-\Ik)=[t-k\delta,t-({k-1})\delta[$$
\be
N^{(l)}_{t-\Ik}=N^{(l)}_{[t-k\delta,t-(k-1)\delta[}=\sum_{\substack{\theta<t,\\ \theta\in\,\mathcal{N}^{(l)}}}\mathds{1}_{\Ik}(t-\theta) \, \, \forall \, t
 \ee
\end{itemize}
\section{Problem}
We want to solve several minimization problems
\be
\min_{a_r\in\R^{(1+MK)}} \dsp\frac{1}{2}a_r^t\,\mathbf{G}\,a_r-b_r^t\,a_r+d_r^t\,|a_r| \ \ \quad \quad (\mathcal{P}_{min}^{(r)})
\ee
where $|v|$ is a vector such that $|v|_i=|v_i|\, \, \forall i$.\\
$b$ and $d$ are matrices of size $(1+MK)$-by-$M$, so, for any $r$, $b^{(r)}$ and $d^{(r)}$ are vectors of $\R^{(1+MK)}$ and are the rth column of $b$ and $d$, respectively.\\
In the following, We use the class \textit{DataSpike}. A \textit{DataSpike} instance has two public attributes:
\begin{itemize}
\item \textit{\_T}: an array of double values containing all spike values independently of the neuron numbers. Values are sorted. \textit{\_T} has only one row.
\item \textit{\_neur}: an array of integer values of the same size of \textit{\_T} containing the neuron numbers.
\end{itemize}
In the algorithms, an instance of the class \textit{DataSpike} is replaced by a matrix of double values with two rows.

\section{Auxiliary functions}
We define the following variables
\begin{itemize}
\item $\alpha$ is the first index of $T$ such that $T[\alpha]> \Tmin$
\item $\beta$ is the last index of $T$ such that $T[\beta]\leq \Tmax$
\item $depart$ is the first index of $T$ such that $T[depart]> \Tmin-A$
\item $departbis$ is the last index of $T$ such that $T[departbis]\leq \Tmax-A$
\end{itemize}
\textit{get\_low\_index} and \textit{get\_up\_index} are functions which define $\alpha$, $\beta$, ...
\begin{itemize}
\item \textit{get\_low\_index} function works as follows:
\be
\hbox{get\_low\_index}(x, T) = \gamma \hbox{ where } \gamma \hbox{ is the first index of T such that }T[\gamma]> x 
\ee
\item \textit{get\_up\_index} function works as follows:
\be
\hbox{get\_up\_index}(x, T) = \gamma \hbox{ where } \gamma \hbox{ is the last index of T such that }T[\gamma]\leq x 
\ee
\end{itemize}

\begin{algorithm}[!h]
\caption{$get\_low\_index$ function}
\begin{algorithmic}[1] 
\Use get\_low\_index(x, T)
  \Desc Computation of ind, the first index in $T$ such that $T[ind]> x$ (to be improved!)
  \Input x; scalar value
\Output ind - integer. If there is no index satisfying the condition, the size of T + 1 is returned.
\State ind $\gets$ 1
\State n $\gets$ length(T)
\While{(ind$<$(n+1))}
\If{(T[ind]$\leq$x)}
\State ind $\gets$ ind+1
\Else
\State break
\EndIf
\EndWhile
\State \Return ind
\end{algorithmic}
\end{algorithm}

\begin{algorithm}[!h]
\caption{$get\_up\_index$ function}
\begin{algorithmic}[1] 
\Use get\_up\_index(x, T)
  \Desc Computation of ind, the last index in $T$ such that $T[ind]\leq x$
  \Input x; scalar value
\Output ind - integer. If there is no index satisfying the condition, 0 is returned.
\State ind $\gets$ length(T)
\While{(ind$\geq$1)}
\If{(T[ind]$>$x)}
\State ind $\gets$ ind-1
\Else
\State break
\EndIf
\EndWhile
\State \Return ind
\end{algorithmic}
\end{algorithm}

\textit{get\_k} function works as follows:
\be
get\_k(x, \delta) = \left\{
\begin{array}{lll}
k&= \hbox{Int}(\dsp\frac{x}{\delta})+1 = \lfloor\dsp\frac{x}{\delta} \rfloor+1 &\hbox{if } {x > k\delta},\\[10pt]
k &= \hbox{Int}(\dsp\frac{x}{\delta}) = \lfloor\dsp\frac{x}{\delta} \rfloor &\hbox{if }{x \equiv k\delta}
\end{array}
\right.
\ee
or $get\_k(x,\delta) = k\, \geq 1$ such that $x \in \Ik=]{(k-1)}\,\delta,\:k\,\delta]$

For example $get\_k(1.5\,\delta, \delta)= 2$ and $get\_k(\delta, \delta)= 1$.

\begin{algorithm}[!h]
\caption{$get\_k$ function}
\begin{algorithmic}[1] 
\Use get\_k(x, delta, eps)
  \Desc Computation of k such that $x \in \Ik=]{(k-1)}\,\delta,\:k\,\delta]$
  \Input x; scalar value
\Input delta; $\delta$
\Input eps; for numerical precision, default value 1e-12 (optional argument)
\Output k; integer
\State k $\gets$ floor(x/delta) + 1
\If{$|k*delta -x|< eps$}
\State \Return k
\EndIf
\If{$|(k-1)*delta -x|< eps$}
\State \Return k-1
\EndIf
\State \Return k
\end{algorithmic}
\end{algorithm}


\section{Computation of b}
\be
b^{(r)} = \left(
\begin{array}{ll}
  &\#^{(r)} = \# \left\{T: \Tmin<T\leq\Tmax,T\in \mathcal{N}^{(r)}\right\}\\
  &\cdots\\
&\dsp\int_{\Tmin}^{\Tmax}N^{(l)}_{[t-k\delta,t-(k-1)\delta[}\,dN^{(r)}_t = \dsp\int_{\Tmin}^{\Tmax}N^{(l)}_{t-\Ik}\,dN^{(r)}_t =\dsp\sum_{\substack{\Tmin<T\leq\Tmax,\\T\in \mathcal{N}^{(r)}}}\,\sum_{\substack{\theta<T,\\ \theta\in\,N^{(l)}}}\mathds{1}_{\Ik}(T-\theta) \\
  &\cdots
\end{array}
\right)
\ee
For $b$ we use also the notation
\be
\mu_1^{(r)} = \left(
\begin{array}{ll}
  &\cdots\\
&\dsp\int_{\Tmin}^{\Tmax}N^{(l)}_{[t-k\delta,t-(k-1)\delta[}\,dN^{(r)}_t = \dsp\int_{\Tmin}^{\Tmax}N^{(l)}_{t-\Ik}\,dN^{(r)}_t \\
  &\cdots
\end{array}
\right)
\ee
\fbox{WARNING}
\begin{itemize}
\item In what follows, numbering starts at 1 (think Matlab/Octave and R) and you should be careful when it differs.
%\item We always assume that no spike is equal to $\Tmin$ or $\Tmax$ for the sake of simplicity.
\item In what follows\\ the loop $for(i= \alpha:\beta)$ is empty if $\alpha> \beta$ (unlike in R)
\end{itemize}



  \begin{algorithm}[!h]
\caption{Computation of $b$}
\begin{algorithmic}[1] 
%\Use compute\_b(M, K, Tmin, Tmax, T, neur, delta)
\Use compute\_b(M, K, Tmin, Tmax, DS, delta)
  \Desc Computation of b - numbering of neurons and k starts at 1
  \Input M; integer - number of neurons
  \Input K; integer - number of bins
\Input Tmin - minimum time
\Input Tmax - maximum time
\Input DS - DataSpike: matrix with two rows, the first one contains spike values (every spikes in a single row), the second one contains the neuron numbers to identify them in the first one
%\Input T - flattened array of spike values (every spikes in a single row)
%\Input neur - array of neuron numbers to identify them in T
\Input delta - $\delta$
\Output b - (1+M*K)-by-M matrix%, as in the Lasso problem $\min 1/2 \, x^t\,A\,x+b^t $ with constraints on $d$
\Output low - array of size Ntot(=length(T)), taking into account the scope
\Output cnt - array of size M, containing $\#^{(r)}$
\State T $\gets$ DS[1,]  \Comment{first row of DS: all spike values}
\State neur $\gets$ DS[2,] \Comment{second row of DS: neuron numbers}
\State Ntot $\gets$ length(T) \Comment{total number of spikes}
\State A $\gets K*\delta$ \Comment{scope}
\State b $\gets$ matrix(0, nrow=(1+M*K), ncol= M) \Comment{init of b}
\State low $\gets$ vector(mode='integer', length=Ntot) \Comment{init of low - for the State}
\State cnt $\gets$ vector(mode='integer', length=M) \Comment{init of cnt - for the algo}
\State ilow $\gets$ 1
\State eps $\gets$ 1.e-12 \Comment{for numerical precision}
\For{(i=1:Ntot)} \Comment{loop over spikes}
\State t $\gets$ T[i]; r $\gets$ neur[i]
\While{$(|T[ilow]-t|>A)$}
\State ilow $\gets$ ilow+1
\EndWhile
\State low[i] $\gets$ ilow
\If{($\Tmin<t\leq\Tmax$)}
\State cnt[r] $\gets$ cnt[r]+1
\For{(j=ilow:(i-1))}
\If{($|t-T[j]|<eps$)}
\State break
%\State next \Comment{same instruction as continue, the spike should be $<t$}
\EndIf
\State k $\gets$ get\_k((t-T[j])/delta)+1 \Comment{more complex than integer part}
\State l $\gets$ neur[j]
%\State b[(r-1)*(K+1)+k+1, r] += 1
\State b[(l-1)*K+k+1, r] $\mathrel{+}=$ 1
\EndFor
\EndIf
\EndFor
\State b[1,]  $\gets$ cnt \Comment{1st line of b}
\State \Return b, low, cnt
\end{algorithmic}
\end{algorithm}
\clearpage
\vspace*{-2cm}
\section{Computation of G}
G is a matrix of size $(1+MK)$-by-$(1+MK)$
\begin{itemize}
  \item{First row or first column of G}
\be
G_{0,l,k}&=&\int_{\Tmin}^{\Tmax}\Nltk\,dt\\
&=& \int_{\Tmin}^{\Tmax}\,\sum_{\substack{\theta<t,\\ \theta\in\,\mathcal{N}^{(l)}}}\mathds{1}_{\Ik}(t-\theta)\,dt\simeq \int_{\Tmin}^{\Tmax}\,\sum_{\substack{(t-A)\leq\theta<t,\\ \theta\in\,\mathcal{N}^{(l)}}}\mathds{1}_{\Ik}(t-\theta)\,dt \\
 &\simeq& \int_{\Tmin}^{\Tmax}\,\sum_{\substack{(t-A)\leq\theta<t,\\ \theta\in\,\mathcal{N}^{(l)}}}\mathds{1}_{\Ik+\theta}(t)\,dt\\
G_{0,l,k}&=&\int_{\Tmin}^{\Tmax}\Nltk\,dt \simeq  \sum_{\substack{\theta\in\,\mathcal{N}^{(l)},\\(\Tmin-A)\leq \theta < \Tmax}}\left( \min{(\Tmax, k\,\delta+\theta)}- \max{(\Tmin, (k-1)\,\delta+\theta)}\right)
\ee
\item{Inner part of G}
\be
G_{l_1,l_2,k_1,k_2}&=&\int_{\Tmin}^{\Tmax}\Nluntkun\,\Nldeuxtkdeux\,dt\\
G_{(l_1,k_1),(l_2,k_2)}&=&\int_{\Tmin}^{\Tmax}\left(\sum_{\substack{\tau\in\mathcal{N}^{(l_1)},\\  \tau< t}}\mathds{1}_{\Ikun+\tau}(t)\right)\,\left(\sum_{\substack{\theta\in\mathcal{N}^{(l_2)},\\\theta<t}}\mathds{1}_{\Ikdeux+\theta}(t)\right)\,dt\\
&\simeq& \int_{\Tmin}^{\Tmax}\left(\sum_{\substack{\tau\in\mathcal{N}^{(l_1)},\\  t-A\leq \tau< t}}\mathds{1}_{\Ikun+\tau}(t)\right)\,\left(\sum_{\substack{\theta\in\mathcal{N}^{(l_2)},\\t-A\leq \theta<t}}\mathds{1}_{\Ikdeux+\theta}(t)\right)\,dt\\
&=& \sum_{\substack{\tau\in\mathcal{N}^{(l_1)},\theta\in\mathcal{N}^{(l_2)},\\  (\Tmin-A)\leq \tau< \Tmax,\\(\Tmin-A)\leq \theta< \Tmax }} \int_{\Tmin}^{\Tmax}\mathds{1}_{\Ikun+\tau}(t)\,\mathds{1}_{\Ikdeux+\theta}(t)\,dt\\
&=& \sum_{\substack{\tau\in\mathcal{N}^{(l_1)},\theta\in\mathcal{N}^{(l_2)},\\  (\Tmin-A)\leq \tau< \Tmax,\\(\Tmin-A)\leq \theta< \Tmax }} \int_{\Tmin}^{\Tmax}\mathds{1}_{(\Ikun+\tau)\cap(\Ikdeux+\theta)\cap]\Tmin,\Tmax]}(t)\,dt
\ee
\end{itemize}
\clearpage
\begin{algorithm}[!h]
\caption{Computation of $G$}
\begin{algorithmic}[1] 
\Use compute\_G(M, K, Tmin, Tmax, T, neur, delta, A)
  \Desc Computation of b - numbering of neurons and k starts at 1
\Input $\delta$; delta 
  \Input M; integer - number of neurons
  \Input K; integer - number of bins
\Input Tmin - minimum time
\Input Tmax - maximum time
\Input T - flattened array of spike values (every spikes in a single row)
\Input neur - array of neuron numbers to identify them in T
\Input delta - size
\Input A - scope, maximum distance to be taken into account
\Output b - array of size (1+M*K)*M%, as in the Lasso problem $\min 1/2 \, x^t\,A\,x+b^t $ with constraints on $d$
\State G $\gets$ matrix(0, nrow=(1+M*K), ncol= (1+M*K)) \Comment{init of G}
\State G[1,1] $\gets \Tmax - \Tmin$
\State A $\gets K*\delta$ \Comment{scope}
\State depart $\gets$ get\_low\_index($(\Tmin-A)$, T)
\State $\beta$ $\gets$ get\_up\_index($\Tmax$, T)
\For{(i in depart:$\beta$)}
\State ti $\gets$ T[i]; $l_1$ $\gets$ neur[i]
\For{($k$ in (1:K))} \Comment{1st row and 1st column of G}
\State $x_1$ $\gets \min{(\Tmax, ti+k\,\delta)}$
\State $x_2$ $\gets \max{(\Tmin, ti+(k-1)\,\delta)}$
\State $dx = x_1 - x_2$
\If{$(dx>0)$}
\State G[1,($l_1$-1)*K+$k$+1] += dx
\State G[($l_1$-1)*K+$k$+1,1] += dx
\EndIf
\EndFor
\For{(j in (low[i]:(i-1)))} \Comment{inner part of G, $l_1 \neq l_2$}
\State tj $\gets$ T[j]; $l_2$ $\gets$ neur[j]
\For{($k_1$ in (1:K))}
\For{($k_2$ in (1:K))}
\State $x_1$ $\gets \min{(\Tmax, ti+k_1\,\delta, tj+k_2\,\delta)}$
\State $x_2$ $\gets \max{(\Tmin, ti+(k_1-1)\,\delta), (tj+(k_2-1)\,\delta)}$
\State $dx = x_1 - x_2$
\If{$(dx>0)$}
\State G[($l_1$-1)*K+$k_1$+1,($l_2$-1)*K+$k_2$+1] += dx
\State G[($l_2$-1)*K+$k_2$+1,($l_1$-1)*K+$k_1$+1] += dx
\EndIf
\EndFor
\EndFor
\EndFor
\For{($k_1$ in 1:K)} \Comment{$l_1==l_2$}
\State $x_1$ $\gets \min{(\Tmax, ti+k_1\,\delta)}$
\State $x_2$ $\gets \max{(\Tmin, ti+(k_1-1)\,\delta)}$
\State $dx = x_1 - x_2$
\If{$(dx> 0)$} \Comment{diagonal part}
\State G[($l_1$-1)*K+$k_1$+1,($l_1$-1)*K+$k_1$+1] += dx
\EndIf
\algstore{algoG}
\end{algorithmic}
\end{algorithm}
\begin{algorithm}[!h]
\begin{algorithmic}[1] 
\algrestore{algoG}
\For{($k_2$ in ($k_1$+1):K)} \Comment{extradiagonal part}
\State $x_2$ $\gets \max{(\Tmin, ti+(k_2-1)\,\delta)}$
\State $dx = x_1 - x_2$
\If{$(dx>0)$}
\State G[($l_1$-1)*K+$k_1$+1,($l_1$-1)*K+$k_2$+1] += dx
\State G[($l_1$-1)*K+$k_2$+1,($l_1$-1)*K+$k_1$+1] += dx
\EndIf
\EndFor
\EndFor
\EndFor
\State \Return G
\end{algorithmic}
\end{algorithm}

\section{Computation of d}
For $d$ we need the following quantities, $\mu_2$ is a $(1+MK)$-by-$M$ matrix.
$\mu_A$ is a vector of dimension $(1+MK)$
\be
\mu_2^{(r)} = \left(
\begin{array}{ll}
  &\cdots\\
&\dsp\int_{\Tmin}^{\Tmax}\left(N^{(l)}_{[t-k\delta,t-(k-1)\delta[}\right)^2\,dN^{(r)}_t = \dsp\int_{\Tmin}^{\Tmax}(N^{(l)}_{t-\Ik})^2\,dN^{(r)}_t \\
  &\cdots
\end{array}
\right)
\ee
\be
\mu_A^{(r)} = \left(
\begin{array}{ll}
  &\cdots\\
&\dsp\sup_{t\in]\Tmin,\Tmax]}|N^{(l)}_{[t-k\delta,t-(k-1)\delta[}|=\dsp\sup_{t\in]\Tmin,\Tmax]}|N^{(l)}_{t-\Ik}|\\
  &\cdots
\end{array}
\right)
\ee



\end{document}

